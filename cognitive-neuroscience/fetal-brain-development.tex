%++++++++++++++++++++++++++++++++++++++++
\documentclass[letterpaper,12pt]{article}
\usepackage{tabularx} % extra features for tabular environment
\usepackage{amsmath}  % improve math presentation
\usepackage{graphicx} % takes care of graphic including machinery
%usepackage{stmaryrd} % for double brackets
%\usepackage[margin=1in,letterpaper]{geometry} % decreases margins
\usepackage{cite} % takes care of citations
\usepackage[final]{hyperref} % adds hyper links inside the generated pdf file
\usepackage{csquotes}
\usepackage{amsthm}
\hypersetup{
	colorlinks=true,       % false: boxed links; true: colored links
	linkcolor=blue,        % color of internal links
	citecolor=blue,        % color of links to bibliography
	filecolor=magenta,     % color of file links
	urlcolor=blue         
}

\newtheorem{theorem}{Theorem}

%++++++++++++++++++++++++++++++++++++++++

\title{Fetal Brain Development}
\author{Atreyee}

\begin{document}

\maketitle

\section{General Takeaway}

Not knowing the technical terms in the paper, the general impression I get from the first page or so is:

\begin{itemize}
\item
The human fetus can exhibit behavioural complexity.
\item
This apparent behavioral complexity comes from the brainstem.
\item
Cognitive and affective processing is mediated by the forebrain.
\item
At birth and for a few weeks after it, the forebrain is still underdeveloped. Thus, we infer that neonatal and immediately post-natal complex behaviour comes from the brainstem.
\end{itemize}

\end{document}