%++++++++++++++++++++++++++++++++++++++++
\documentclass[letterpaper,12pt]{article}
\usepackage{tabularx} % extra features for tabular environment
\usepackage{amsmath}  % improve math presentation
\usepackage{graphicx} % takes care of graphic including machinery
%usepackage{stmaryrd} % for double brackets
%\usepackage[margin=1in,letterpaper]{geometry} % decreases margins
\usepackage{cite} % takes care of citations
\usepackage[final]{hyperref} % adds hyper links inside the generated pdf file
\usepackage{csquotes}
\usepackage{amsthm}
\hypersetup{
	colorlinks=true,       % false: boxed links; true: colored links
	linkcolor=blue,        % color of internal links
	citecolor=blue,        % color of links to bibliography
	filecolor=magenta,     % color of file links
	urlcolor=blue         
}

\newtheorem{theorem}{Theorem}

%++++++++++++++++++++++++++++++++++++++++

\title{Neuron Structure And Function}
\author{Atreyee}

\begin{document}

\maketitle

\section{Human Nervous System}

\subsection{Central Nervous System}

\subsection{Peripheral Nervous System}

\section{Types of Neurons}

\subsection{Sensory}

\subsection{Motor}

\subsection{Interneurons}

An example of specialization within neurons:

\subsubsection{Purkinje Cells}

\section{Neuron: Functions}

Neurons form networks to fulfill the nervous system's more complex functions.

\begin{itemize}
\item
Receive information
\item
Process/integrate information
\item
Communicate/send information
\end{itemize}

\section{Components of a Neuron}

\subsection{Dendrites}

Action potentials:

\subsubsection{Excitatory}
\subsubsection{Inhibitory}

\subsection{Axon}
\subsection{Myelin}
\subsection{AxonTerminal}
\subsection{Synapse}

\section{Glial Cells}

\subsection{Astrocytes}

\subsection{Microglia}

\subsection{Oligodencrocytes}

CNS

\subsection{Schwann Cells}

PNS

\subsection{Satellite Cells}

Not too well understood.

\subsection{Ependymal Cells}

The ones with the cilia


\end{document}