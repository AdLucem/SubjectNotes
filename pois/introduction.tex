%++++++++++++++++++++++++++++++++++++++++
\documentclass[letterpaper,12pt]{article}
\usepackage{tabularx} % extra features for tabular environment
\usepackage{amsmath}  % improve math presentation
\usepackage{graphicx} % takes care of graphic including machinery
%usepackage{stmaryrd} % for double brackets
%\usepackage[margin=1in,letterpaper]{geometry} % decreases margins
\usepackage{cite} % takes care of citations
\usepackage[final]{hyperref} % adds hyper links inside the generated pdf file
\usepackage{csquotes}
\usepackage{amsthm}
\hypersetup{
	colorlinks=true,       % false: boxed links; true: colored links
	linkcolor=blue,        % color of internal links
	citecolor=blue,        % color of links to bibliography
	filecolor=magenta,     % color of file links
	urlcolor=blue         
}

\newtheorem{theorem}{Theorem}

%++++++++++++++++++++++++++++++++++++++++

\title{POIS: Impossible Problems, Imperfect Solutions}
\author{Atreyee}

\begin{document}

\maketitle

\begin{displayquote}

\textit{All information security problems have a common feature, which is that they are impossible to solve perfectly.}

\end{displayquote}

\begin{displayquote}

\textit{All information security problems have a common solution and that is the use of destructive interference of impossibilities.}

\end{displayquote}

\begin{displayquote}

\textit{Therefore, this course is truly fundamental and should be named The Science Of The Impossible.}

- Prof. Kannan Srinathan 
\end{displayquote}

\section{Principles of Security}

\subsection{Kirchoff's Principle of Security}

Security is derived from the secrecy of the encryption key, not from the obscurity of the encryption algorithm. 

Thus, an ideal hashing algorithm should be non-invertible.

\subsection{Principle of Sufficiently Large Keyspace}

\section{Ciphers}

\subsection{Definition}

\subsection{Monoalphabetic Substitution Cipher}

How to break: frequency attack

\subsection{Vigeneve Cipher}

How to break: TODO

\section{The Secrets of Secrecy}

\subsection{Heuristic Secrecy}
\subsection{Provable Secrecy}
\subsection{Proven Secrecy}
\subsection{Shannon's Perfect Secrecy}

\section{The Perfect Magical Cipher: Vernam's Cipher}

\begin{theorem}
One Time Pad is perfectly secret.
\end{theorem}

\begin{proof}
TODO: FIND PROOF
\end{proof}

\section{We Need More Keys}

\begin{theorem}
The condition of perfect secrecy is that the keyspace should be larger than or equal to the message space.
\end{theorem}

\begin{proof}
TODO: GET PROOF
\end{proof}

\section{Channeling Secrecy}

\begin{theorem}
Transmission of a perfectly secure message can only be done at the rate of the secure channel.
\end{theorem}

\begin{proof}
TODO: GET PROOF
\end{proof}

\end{document}