\documentclass[a4paper]{article}

\usepackage[english]{babel}
\usepackage[utf8]{inputenc}
\usepackage{amsmath}
\usepackage{graphicx}
\usepackage[colorinlistoftodos]{todonotes}

\title{Morphology of Hyderabadi Dakhini Urdu}

\author{Atreyee Ghosal: 20161167}

\date{\today}

\begin{document}
\maketitle

\begin{abstract}
Enter a short summary here. What topic do you want to investigate and why? What experiment did you perform? What were your main results and conclusion?
\end{abstract}

\section{Nouns}

\subsection{Three Classes of Nouns}

\subsubsection{Class 1: Marked Masculine Noun}

\begin{itemize}
\item
This type of noun terminates in the suffixes:
	\begin{itemize}
	\item
	\emph{-aa} in the nominative singular
	\item
	\emph{-e} in the nominative plural and in the oblique singular
	\end{itemize}
\item[Eg:  ]
	\begin{itemize}
	\item[child]
	baccha (singular), bacche (plural)
	\item[cloth]
	kapRaa (singular), kapRe (plural)
	\end{itemize}
\end{itemize}

\subsubsection{Unmarked Masculine/Feminine Noun}

\begin{itemize}
\item 
These nouns occur with the plural suffix \emph{-aaN}
\item[Eg:  ]
	\begin{itemize}
	\item[apple]
	seb (singular), sebaaN (plural)
	\item[book]
	kitaab (singular), kitabaaN (plural) 
	\end{itemize}
\end{itemize}

\subsubsection{Unmarked Masculine/Feminine}

\begin{itemize}
\item
These nouns occur with a zero plural suffix.
\item[Eg:  ]
	\begin{itemize}
	\item[aunts]
	xaalaa (singular), xaalaa(plural)
	\item[mango]
	aam (singular), aam (plural)
	\end{itemize}
\end{itemize}

\subsection{Postposition Class}

The postposition class is the class of, well, postpositions. This class influences nouns as all nouns occur with oblique-case suffixes before all members of this class. This is a closed class of functional morphemes, containing:

\begin{itemize}
\item
kaa/kii/ke, 'of'
\item
kuu/ko, 'to'
\item
meN, 'in'
\item
pe, 'on'
\item
tak, 'until'
\end{itemize}

\subsection{Variations}

 The below are the variations shown in the noun class between the different 'systems' of Dakhini rules.
 
\subsubsection{Weakening of Gender Distinction}

Unmarked feminine nouns and loan words are often regarded as maasculine in gender, even though this may not be grammatically correct within the main system.

\subsubsection{Assimilation into Class 3}

A number of nouns that belong to classes 1 and 2 are treated by speakers as if they belong to class 3, deriving their plurals with the addition of a zero suffix.

\subsubsection{Interference from Standard Urdu}

Standard urdu contains a practice where feminine nouns are classed separately from masculine nouns using suffixes. Dakhini speakers who have studied standard Urdu in schools occasionally attempt to use these suffixes in speech.

\section{Adjectives}

Adjectives fall into two classes: Marked and Unmarked.

\subsection{Marked}

\begin{itemize}
\item
Marked adjectives terminate in \emph{-aa/-e/-ii}.
\item
Marked adjectives agree in gender, number and case with the nouns they modify.
\item
This pattern of marked adjectives is similar to that found in standard urdu.
\end{itemize}

\subsection{Unmarked}

Unmarked adjectives do not agree with the nouns that they modify, and do not have any regular morphology.

\section{Personal Pronouns}

In free variation; however no single speaker's idiolect includes all the forms of pronouns listed.

\subsection{First Person}

\begin{itemize}
\item
Nominative: maiN
\item
Oblique: muje
\end{itemize}

\subsection{Second Person}

\begin{itemize}
\item
Nominative: tuu
\item
Oblique: tuje
\end{itemize}

\subsection{Third Person}

\begin{itemize}
\item
Proximate Nominative: ine/ye
\item
Proximate Oblique: ise/is
\item
Distant Nominative: une/wo
\item
Distant Oblique: use/us
\end{itemize}

The existence of two types of third person pronouns is hypothesized to be because of the disappearance of the \emph{Agentive Construction} in the past tense - i.e: standard urdu \textit{ne} --> \textit{ine, une}

\textit{hameN, tumheN} --> the usage of these as nominative rather than oblique forms appears to be a Dakhini innovation



\section{Relative and Interrogative Pronouns}

Data concerning relative and interrogative pronouns is less, but we present the results that could be gathered.

The following interrogative pronouns occur in the Dakhini corpus: jo, kaun and kis kuu.

\subsection{Reflexive Adjective}

The reflexive adjective \emph{apnaa/apnii/apne} occurs.

\subsection{Variations}

\begin{itemize}
\item
Closer To The Paradigm = Wider Variety: speakers whose idiolects approach most closely to the primary system of Dakhini Urdu employ the widest variety of pronouns.
\item
Substitution of Plural for Singular: \emph{ham} is sometimes substituted for \emph{maiN}
\item
Interference From Standard Urdu: the standard Urdu second person honorific pronoun \emph{aap} is sometimes used by Dakhini speakers, particularly by those who have studied standard Urdu in school.
\end{itemize}

\section{Verbs}

Four types of verb constructions:

\subsection{Formed From Present Participle}

\begin{itemize}
\item
The present participle of a verb is formed by the suffixation of \emph{-taa/-tii/-tee} to the verb root.
\item
Participles agree in gender and number with the subject of the sentence.
\end{itemize}

\subsubsection{Present Participle = Indefinite Tense}

\begin{itemize}
\item
The present participle occurs as the indefinite tense with the auxiliary 'hai' in declarative sentences.
\item
The present participle occurs as the indefinite tense without the auxiliary in interrogative and negative sentences
\item[Eg:   ]
	\begin{itemize}
	\item
	main jata hoon \textit{I go}
	\item
	tu kya jata \textit{you (interrogative marker) go}
	\end{itemize}
\end{itemize}

\subsubsection{Present Participle + Past Auxiliary = Imperfect Tense}

\begin{itemize}
\item
An imperfect tense is formed from the present participle with the past auxiliary \emph{thaa/thii/the}
\item
Example: ghar pe saawkaraaN aate the \textit{The merchants used to come to our house}
\end{itemize}

\subsubsection{Adverbial: Present Participle ending in -te}

\begin{itemize}
\item
The present participle terminating in \emph{-te} is used adverbially to indicate duration or to describe the context of an action.
\end{itemize}


\subsubsection{Variations}

\begin{itemize}
\item
Non-Use of the auxiliary hai in indefinite tense statements
\item
Disappearance of the feminine plural suffix
\item
Interference from standard urdu: occasionally the free form of the auxiliary occurs after present participles in Dakhini.
\end{itemize}

\subsection{Formed From Past Participle}

\begin{itemize}
\item
Past participle is formed from the suffixing of \emph{-aa/-ii/-a/-iiN} to the verb root.
\item
Past participle agrees with gender and number of the subject noun.
\end{itemize}

\subsubsection{Past Participle = Past Tense}

\begin{itemize}
\item
The past tense can be formed by the past participle either occurring alone
\item
Or occurring with the auxiliary 'hai' bound to the verb stem.
\end{itemize}

\subsubsection{Past Participle + Past Auxiliary = Pluperfect Tense}

\begin{itemize}
\item
A pluperfect tense is formed from the past participle with the past auxiliary \emph{thaa/thii/the}
\end{itemize}

\subsubsection{Adverbial: Past Participle ending in -e}

Past participles terminating in -e are used adverbially to denote:

\begin{itemize}
\item
Context of an action
\item
Followed by 'so' to denote the context of an action
\item
Followed by 'tak', meaning 'until'
\end{itemize}

\subsubsection{Variations}

\begin{itemize}
\item
Non-use of the auxiliary \textit{hai} in past tense constructions
\item
Disappearance of the feminine plural form
\end{itemize}

\subsection{Based on Verb Root}


\subsubsection{Root + raa + hai = Continuous Present Tense}

\begin{itemize}
\item
The continuous present tense is constructed from the verb root and a postposition auxiliary \emph{raa/rii/re/rai}, suffixed with a bound form of the auxiliary 'hai'
\end{itemize}

\subsubsection{Root + raa + thaa = Past Continuous Tense}

\begin{itemize}
\item
The continuous past tense is constructed from the verb root and a postposition auxiliary \emph{raa/rii/re/rai}, suffixed with a bound form of the auxiliary \emph{thaa/thii/the}
\end{itemize}

\subsubsection{Adverbial Construction: Special}

The root verb itself is used as an adverbial to denote the last of any sequence of actions occurring in a consequent time context.

\subsubsection{Variations}

\begin{itemize}
\item
Non-use of the auxiliary \textit{hai} in the formation of the present continuous
\item
Two alternate forms of regular primary system constructions are found
\item
Interference from standard urdu: a number of people occasionally use the forms \emph{rhaa/rhii/rhe} in place of \emph{raa/rii/re}, and skip the ending auxiliary.
\item
Variant forms of \textit{ko}
\end{itemize}

\subsection{Verb Root + Inflectional Suffixes, Infinitives}

\subsubsection{Subjunctive}

\begin{itemize}
\item
The subjunctive form is formed exactly as in standard urdu
\item
The plural form of the subjunctive, ending in -eN, is not found in Dakhini
\end{itemize}

\subsubsection{Verb root + zero suffix = imperative}

The pronoun 'tuu', plus an infinite form of the verb with zero sffix, forms the imperative.

\subsubsection{Subjunctive + gaa/gii = Future Tense}

The future tense is formed by the addition of \emph{-gaa/-gii/-ge} to the base provided by the subjunctive form.

\subsubsection{Verb Root + naa = Infinitive}

The infinitive is formed by the addition of \emph{-naa} to the verb root.

\subsubsection{Variations}

\begin{itemize}
\item
Substitution of the infinitive for the first person singular subjunctive
\item
Non-use of the future tense terminating in \textit{gaa}
\item
Formation of the infinitive by the suffixation of \textit{-aunaa} rather than \textit{naa}
\end{itemize}


\section{The Primary System of Dakhini Urdu}

The primary system is considered to be that version of the dialect which:

\begin{itemize}
\item
Contains a maximum number of historically derived distinctions
\item
Contains a minimum number of innovations in the use of historical forms
\item
Contains a minimum number of forms or constructions borrowed from other languages
\end{itemize}

\subsection{Features of Primary Dakhini}

Based on the collected data and interviews with the speaker, we can determine that the speaker of a primary system version of Dakhini has the following features in their speech:

\begin{itemize}
\item
Future tense terminates in either \textit{-gaa}, \textit{-gii} or \textit{-ge}
\item
The subjunctive first person singular terminates in \textit{-uuN}
\item
In statements containing the indefinite tense, the present participle + auxiliary construction is used to make up the indefinite tense of the verb
\item
In the past tense, the past participle + auxiliary construction is occasionally used instead of just the past participle
\item
The auxiliary is used in the formation of the present continuous tense
\item
A complete pronoun paradigm- i.e all the pronouns in all the classes mentioned are used.
\item
The feminine third person plural ends in \textit{-ain} or \textit{-iin}.
\item
Correct gender classification of nouns into their respective masculine/feminine forms
\end{itemize}

\subsection{Omissions in Primary Dakhini}

Based on the data collected, the following features are definitely not present in the speech of a speaker of the primary system version of Dakhini:

\begin{itemize}
\item
Substitution of the first person plural pronoun \textit{ham} in place of the first person singular pronoun \textit{maiN}
\item
Substitution of the masculine suffix for the feminine suffix in second person plural constructions.
\end{itemize}
\end{document}